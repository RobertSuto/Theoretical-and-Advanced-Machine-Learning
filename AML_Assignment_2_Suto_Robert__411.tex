\documentclass[11pt, a4paper]{article}
\usepackage[utf8]{inputenc}
\usepackage[skip=5pt plus1pt, indent=15pt]{parskip}
\usepackage{setspace}
\onehalfspacing
\usepackage{unicode-math}
\usepackage{amsmath}
\usepackage{geometry}
\usepackage{bbold}
\usepackage{appendix}
\usepackage{hyperref}
\usepackage{xcolor}
\usepackage{graphicx}
\usepackage{hyperref}
\usepackage[export]{adjustbox}[2011/08/13]
\usepackage{url}
\usepackage{float}
\usepackage{indentfirst}
\usepackage{caption}
\usepackage{subcaption}
\usepackage[shortlabels]{enumitem}
\hypersetup{
    colorlinks=true,
    linkcolor=blue,
    filecolor=magenta,      
    urlcolor=cyan,
    pdftitle={Overleaf Example},
    pdfpagemode=FullScreen,
    }
\usepackage{graphicx}
\usepackage{etoolbox,refcount}
\usepackage{multicol}

\newcounter{countitems}
\newcounter{nextitemizecount}
\newcommand{\setupcountitems}{%
  \stepcounter{nextitemizecount}%
  \setcounter{countitems}{0}%
  \preto\item{\stepcounter{countitems}}%
}
\makeatletter
\newcommand{\computecountitems}{%
  \edef\@currentlabel{\number\c@countitems}%
  \label{countitems@\number\numexpr\value{nextitemizecount}-1\relax}%
}
\newcommand{\nextitemizecount}{%
  \getrefnumber{countitems@\number\c@nextitemizecount}%
}
\newcommand{\previtemizecount}{%
  \getrefnumber{countitems@\number\numexpr\value{nextitemizecount}-1\relax}%
}
\makeatother    
\newenvironment{AutoMultiColItemize}{%
\ifnumcomp{\nextitemizecount}{>}{3}{\begin{multicols}{2}}{}%
\setupcountitems\begin{itemize}}%
{\end{itemize}%
\unskip\computecountitems\ifnumcomp{\previtemizecount}{>}{3}{\end{multicols}}{}}


\title{AML Assignment 2}
\author{Suto Robert-Lucian (411)}
\date{}

\begin{document}

\maketitle
\begin{large}

\section{Exercise 1.}
Consider $\mathcal{H}$ the following hypothesis class :
$$
\mathcal{H}=\left\{h_a: \mathbb{R} \rightarrow\{0,1\} \mid a>0, a \in \mathbb{R}, \text { where } h_a(x)=\mathbf{1}_{[-a, a]}(x)=\left\{\begin{array}{ll}
1, & x \in[-a, a] \\
0, & x \notin[-a, a]
\end{array}\right\}\right.
$$
\subsection{a. Compute the growth function $\tau_H(m)$ (also known as shatter coefficient) for $m \geq 0$ for hypothesis class $\mathcal{H}$}

From a first look, we can see that our hypothesis class, has a similar form to $\mathcal{H}_{thresholds}$, thus we can have a look at Lecture 8.
\\

Consider $\mathrm{C}=\left\{\mathrm{c}_1, \mathrm{c}_2, \ldots, \mathrm{c}_{\mathrm{m}}\right\}$ a set of $\mathrm{m}$ points, with $\mathrm{c}_{\mathrm{i}}<\mathrm{c}_{\mathrm{j}}$. \\
$\mathcal{H}_C$ can have at most $\mathrm{m}+1$ different functions: take $\mathrm{a}_1<\mathrm{c}_1<\mathrm{a}_2<\mathrm{c}_2<\ldots<\mathrm{a}_{\mathrm{m}}$ $<\mathrm{c}_{\mathrm{m}}<\mathrm{a}_{\mathrm{m}+1}$, than we will have $\left|\mathcal{H}_C\right|=\left\{\mathrm{h}_{\mathrm{a}_{-} 1}, \mathrm{~h}_{\mathrm{a}_2 2}, \ldots, \mathrm{h}_{\mathrm{a}_{-} \mathrm{m}+1}\right\}=\mathrm{m}+1$ as\\
$\mathrm{h}_{\mathrm{a} \_1}$ labels points $\mathrm{c}_1, \mathrm{c}_2, \ldots, \mathrm{c}_{\mathrm{m}}$ with labels $(0,0,0, \ldots, 0,0)$\\
$\mathrm{h}_{\mathrm{a} \_2}$ labels points $\mathrm{c}_1, \mathrm{c}_2, \ldots, \mathrm{c}_{\mathrm{m}}$ with labels $(1,0,0, \ldots, 0,0)$\\
$\ldots\ldots\ldots\ldots\ldots\ldots\ldots\ldots\ldots\ldots\ldots\ldots
\ldots\ldots\ldots\ldots\ldots\ldots\ldots$\\
$\mathrm{h}_{\mathrm{a} \_3}$ label points $\mathrm{c}_1, \mathrm{c}_2, \ldots, \mathrm{c}_{\mathrm{m}}$ with labels $(1,1,0, \ldots, 0,0)$\\
$\mathrm{h}_{\mathrm{a} \_\mathrm{m}}$ label points $\mathrm{c}_1, \mathrm{c}_2, \ldots, \mathrm{c}_{\mathrm{m}}$ with labels $(1,1,1, \ldots, 1,0)$\\
$\mathrm{h}_{\mathrm{a} \_\mathrm{m}+1}$ label points $\mathrm{c}_1, \mathrm{c}_2, \ldots, \mathrm{c}_{\mathrm{m}}$ with labels $(1,1,1, \ldots, 1,1)$
\\

Which concludes that:
$$
\tau_H(m)=\max _{C \subseteq X: C \mid=m}\left|H_C\right| = m+1
$$


\subsection{b. Compare your result from the previous point with the general upper bound given by the Sauer lemma. Are they equal or different?}
In order to compute the upper bound, given by the Sauer lemma, we need to calculate the VCdim($\mathcal{H}$). We can simply prove that the VCdim($\mathcal{H}$) = 1.

Consider $x_1,x_2$, two points. While placing them inside our hypothesis class, we can observe that we have three possible cases:
\begin{itemize}
    \item One, where $|x_1| < |x_2|$, where the label (0,1) cannot be achieved;
    \item Second one, where $|x_1| = |x_2|$. The problem is the labels (1,0) and (0,1) cannot be achieved;
    \item And the last one, $|x_1| > |x_2|$, where the label (1,0) cannot be achieved;
\end{itemize}
Because two points cannot be shattered, we can safely say that VCdim($\mathcal{H}$) = 1.

According to Sauer's lemma, the upper bound is given by the following inequality:
$$
\tau_{\mathcal{H}}(m) \leq \sum_{i=0}^d C_m^d
$$
where $d$ is given by VCdim($\mathcal{H}$), and is known to be 1. Thus, the upper bound, given by the lemma is:
$$
\sum_{i=0}^d C_m^i=C_m^0+C_m^1=1+m
$$
which is equal to what we computed at subpoint a.
\newpage

\section{Exercise 2.}
Consider de concept class $C_a$ formed by the union of two closed intervals $[a, a+1] \cup[a+2, a+4]$, where $a \in \mathbb{R}$. Give an efficient ERM algorithm for learning the concept class $C_a$ and compute its complexity for each of the following cases:
\subsection{a. realizable case.}
\subsection{b. agnostic case.}


\section{Exercise 3.}
Consider a modified version of the AdaBoost algorithm that runs for exactlythree rounds as follows:
\begin{itemize}
\item the first two rounds run exactly as in AdaBoost (at round 1 we obtain distribution $\mathbf{D}^{(1)}$, weak classifier $h_1$ with error $c_1$; at round 2 we obtain distribution $\mathbf{D}^{(2)}$, weak classifier $h_2$ with error $\left.c_2\right)$
\item in the third round we compute for each $i=1,2, \ldots, m$ :
$$
\mathbf{D}^{(3)}(i)= \begin{cases}\frac{D^{(1)}(i)}{Z}, & \text { if } h_1\left(x_i\right) \neq h_2\left(x_i\right) \\ 0, & \text { otherwise }\end{cases}
$$
where $\mathrm{Z}$ is a normalization factor such that $\mathbf{D}^{(3)}$ is a probability distribution.
\item obtain weak classifier $h_3$ with error $\epsilon_3$.
\item output the final classifier $h_{f i n a l}
(x)=\operatorname{sign}\left(h_1(x)+h_2(x)+h_3(x)\right)$.
\end{itemize}
Assume that at each round $t=1,2,3$ the weak learner returns a weak classifier $h_t$ for which the error $c_t$ satisfies $c_t \leq \frac{1}{2}-\gamma_t, \gamma_t>0$.
\\
\subsection{a) What is the probability that the classifier $h_1$ will be selected again at round 2 ? Justify your answer.}

\subsection{b) Consider $\gamma=\min \left\{\gamma_1, \gamma_2, \gamma_3\right\}$. Show that the training error of the final classifier $h_{\text {final }}$ is at most $\frac{1}{2}-\frac{3}{2} \gamma+2 \gamma^3$ and show that this is strictly smaller than $\frac{1}{2}-\gamma \cdot$}
$$
\begin{aligned}
& \frac{1}{2}-\frac{3}{2} \gamma_{\min }+2 \gamma_{\min }^3<\frac{1}{2}-\gamma_{\min } \implies \\
& -\frac{1}{2} + \gamma_{\min } + \frac{1}{2}-\frac{3}{2} \gamma_{\min }+2 \gamma_{\min }^3<0 \implies \\
& -\frac{3}{2}  \gamma_{\min }+2  \gamma_{\min }^3+\gamma_{\min }<0 \implies \\
& \frac{3}{2}  \gamma_{\min }-2  \gamma_{\min }^3-\gamma_{\min }>0 \implies \\
& \frac{1}{2}  \gamma_{\min }-2  \gamma_{\min }^3>0 \implies \\
& \gamma_{\min }\left(\frac{1}{2} -2 \gamma_{\min }^2\right)>0 \implies  \\
& \frac{1}{2} -2 \gamma_{\min }^2>0 \implies  \\
& 2 \gamma_{\min }^2 < \frac{1}{2} \implies  \\
& \gamma_{\min }^2 < \frac{1}{4} \implies  \\
& \gamma_{\min } < \frac{1}{2} \\
\end{aligned}
$$



From the assumption made in the request (the weak learner returns a weak classifier $h_t$ for which the error $c_t$ satisfies $c_t \leq \frac{1}{2}-\gamma_t, \gamma_t>0$), we can affirm that $\gamma_{\min } < \frac{1}{2}$ is true, hence,  is correct.


\end{large}
\end{document}
